\documentclass[12pt]{article}%
\usepackage{amssymb}
\usepackage{amsfonts}
\usepackage{amsmath}
\usepackage[nohead]{geometry}
\usepackage[singlespacing]{setspace}
\usepackage[bottom]{footmisc}
\usepackage{indentfirst}
%\usepackage[danish]{babel}
\usepackage{endnotes}
\usepackage{graphicx}%
\usepackage{rotating}
\usepackage[hypertex]{hyperref}
\setcounter{MaxMatrixCols}{30}
\newtheorem{theorem}{Theorem}
\newtheorem{acknowledgement}{Acknowledgement}
\newtheorem{algorithm}[theorem]{Algorithm}
\newtheorem{axiom}[theorem]{Axiom}
\newtheorem{case}[theorem]{Case}
\newtheorem{claim}[theorem]{Claim}
\newtheorem{conclusion}[theorem]{Conclusion}
\newtheorem{condition}[theorem]{Condition}
\newtheorem{conjecture}[theorem]{Conjecture}
\newtheorem{corollary}[theorem]{Corollary}
\newtheorem{criterion}[theorem]{Criterion}
\newtheorem{definition}[theorem]{Definition}
\newtheorem{example}[theorem]{Example}
\newtheorem{exercise}[theorem]{Exercise}
\newtheorem{lemma}[theorem]{Lemma}
\newtheorem{notation}[theorem]{Notation}
\newtheorem{problem}[theorem]{Problem}
\newtheorem{proposition}{Proposition}
\newtheorem{remark}[theorem]{Remark}
\newtheorem{solution}[theorem]{Solution}
\newtheorem{summary}[theorem]{Summary}
\newenvironment{proof}[1][Proof]{\noindent\textbf{#1.} }{\ \rule{0.5em}{0.5em}}
\makeatletter
\def\@biblabel#1{\hspace*{-\labelsep}}
\makeatother
\geometry{left=0.8in,right=0.8in,top=1.00in,bottom=1.0in}
\begin{document}

\title{Political Institutions and Economic Development }
\author{Sebastian Barfort\medskip\\{\normalsize} \date{\normalsize \today}}
\maketitle

\sloppy



%\textbf{Keywords:} LaTeX; economics research.

\strut

%\textbf{JEL Classification Numbers:} Y90 (Miscellaneous Categories -Other -Other).

\thispagestyle{empty}

\pagebreak%
%\onehalfspacing

\singlespacing

\section{}
At least two problems prevent clear identification of the effect of institutions on economic growth: first, we lack a clear understanding of what constitutes an ``institution.'' Relying on North's (1990) ``humanly devised constraints'' definition of formal and informal rules does not give us much in terms of meaningful empirical identification. Second, we have no accepted theory of endogenous institutional change, and thus no unified growth model in which we can evaluate different institutions. In a recent paper, Jones and Romer (2009, p. 4) note that 
\begin{quote}
we follow the example of the neoclassical model and treat institutions the way the neoclassical model treated technology, as an important force that enters the formalism but which evolves according to a dynamic that is not explicitly modelled. Out on the horizon, we can expect that current research on the dynamics of institutions and politics will ultimately lead to a simple formal representation of endogenous institutional dynamics as well.
\end{quote}

The following section is heavily constrained by the imprecise definition and lack of a formal model, and it follows that any evaluation of different institutions will be somewhat arbitrary. I proceed in the following way: first, I present some (perhaps) competing theories that link different institutions to growth.\footnote{Note that due to the very binding space constraint I exclude some prominent theories such as political accountability. Furthermore, I do not mention Engerman and Sokoloff's theories on the effect of inequality. This is of course unfortunate, but nonetheless necessary.} Second, I describe two interesting new empirical approaches. Third, I present a new approach, inspired by ongoing work by Romer (2010), of thinking about rules and meta-rules rather than institutions and argue that rules might be more likely to be econometrically identified and that this route represents an interesting path for future empirical work. 

Most economists and political scientists would agree today that ``institutions matter'' for economic growth. But which institutions? Undoubtedly most widespread is the belief, pioneered by Jones (1981) and elaborated by North (1991) and Acemoglu and others (1998, 2002, 2005), that advance property rights institutions to the center of the debate. The idea is intuitive and fits neatly into a standard endogenous growth framework; the decision of an agent to invest in physical or human capital depends negatively on the risk that their gains are expropriated by the government. When property rights institutions fail to constrain those who control the state the payoff of the investment is discounted at a higher rate, which lovers investment and thus steady state growth. Several empirical papers seem to confirm a link between property rights protection and economic growth (AJR 2002, Banerjee and Iyer 2005, Dell 2010), but it has not been convincingly demonstrated that results are not affected through alternative channels (Keefer 2004). 

A competing theory, pioneered by the New Institutional Economics framework developed by Williamson (1975, 1985) and others, underscores the importance of contracting institutions. According to this idea economic growth is intimately linked to what type of contracts can be written and enforced in the economy. Although there might be large overlaps between these theories, they differ in some important aspects. Primarily, property rights theorists put more emphasis on the distribution of political power. Although attempts have been made (Acemoglu 2006), these differences might be very difficult to identify econometrically, since there is a tendency for institutions to cluster. That is, countries with weak property rights tend to also be countries with weak contracting institutions. 

Przeworski (2010) notes that property rights are themselves created by the state, making it difficult to believe that property rights are the ``fundamental'' cause of economic growth. One obvious question one need to worry about is the endogeneity of property rights; property rights are enforced by the agents who control the state at a given time, but how are the rules that govern the rulers decided? Furthermore, since property rights are a bundle of rights, and it is not clear how these different elements in the set relate to growth.  \\

As already mentioned, there is a large empirical literature that try to disentangle the effects of different institutions, but so far without much convincing evidence in any direction. Two recent studies provide some grounds for optimism: Fearon, Humphreys and Weinstein (2011) and Casey, Glennerster and Miguel (2011) study institutional change in Liberia and Sierra Leone, respectively. Both papers analyze the effects of (exogenously) making local institutions more egalitarian and democratic on the ability of the communities to solve collective action problems. Both studies rely on randomized field experiments and use sophisticated measures from the experimental economics literature of collective action provision. Even though they find completely opposite results on economic outcomes this approach seems like an interesting way forward. 

As long as our theoretical notion of an institution remains weak identification of causal effects seems very unlikely. In ongoing work, Romer (2010) suggests breaking our concept of institutions down into thinking about rules and meta-rules. An institution is a set of rules and meta-rules. Rules specify how people interact with each other. An example of a rule can be whether or not private companies are allowed to operate in a market. The relation to standard growth theory is straightforward: total factor productivity depends on the stock of human capital and local rules. With different rules, different technologies maximizes productivity. With different stock of technology, different rules will be efficient. According to Romer, different rules can prevent or exacerbate the adoption of new technology. However, we need to consider the endogeneity of these rules. To accommodate this endogeneity, Romer defines meta-rules, the rules for changing rules. An obvious example of a meta-rule is democracy. 

I would argue that the attempt to break the notion of an institution down into a bundle of rules and meta-rules represents a promising way forward that fits well with current experimental work studying the effect of local changes at the community level using microeconometric techniques. History has not provided us with many (if any) examples of exogenous changes in institutions, and looking, perhaps less ambitiously, for variation in different rules or meta-rules seems much like a much more promising strategy for obtaining unbiased estimates of the effects on economic outcomes such as growth. 

\section{}
Much work in applied economics and political science contain subtle arguments and complicated statistical methods in attempts to establish statistical causation without devoting much space to other important aspects such as a philosophical or semantic understanding of the concept. However, we can only begin to investigate statistical causation after having established a clear understanding of the semantics of such a description. My comparative advantage lies elsewhere, and I have very few insightful comments to add to Dawid's (2000, p. 407) understanding. Thus, the rest of this paper assumes that we share a very clear semantic definition of the concept of causality. 

This section proceeds in the following way: to fix ideas, some space is devoted to a standard textbook approach to thinking about statistical causation. I proceed by discussing a general problem of thinking about counterfactuals, and the validity of the random assignment and constant effects assumptions that I consider the two most problematic assumptions in identifying causal effects of institutions on growth. I try to give examples of applied work whenever possible. 

A simplified textbook approach to causality looks something like the following:\footnote{Notation as in Angriest and Pischke (2009)} for simplicity, assume our institutional treatment variable, $D$, can only take two values: $D_i \in \{0,1\}$ for all $i$. Thus, we can write the potential outcome for individual/municipality/nation (from now on: individual) $i$. For each individual in the population we observe either $Y_{1i}$, or $Y_{0i}$, bot never both. Thus, we need to engage in a thought experiment; we simply ask ourselves what the effect would have been on individual $i$ had she not been treated, and vice versa. The following highlights the counterfactual nature of a causal effect: Because statistical inference requires a certain quantity of observations, it is impossible to measure causal effects at the individual level. Researchers therefore focus on average causal effects. The two most frequently used in the literature is the Average Treatment Effect (ATE): $E[Y_{1i}-Y_{0i}]$ and the Average Treatment Effect on the Treated (ATET): $E[Y_{1i}-Y_{0i}
|D_i=1 ]$, where $E[\cdot]$ is the well known expectations operator. Note that the latter can be rewritten as
\begin{equation}
E[Y_{1i}-Y_{0i}|D_i=1 ]=E[Y_{1i}
|D_i=1 ]-E[Y_{0i}
|D_i=1 ].
\end{equation}
Equation (2) illustrates the counterfactual aspect of causal inference: The first term captures the expected value of the treated, whereas the second term asks what the expected value of the treated would be \textit{had they not been treated}. The fact that the counterfactual term is (by definition) never observed has caused a lively debate about the validity of causal inference among statisticians, most prominently represented by Dawid's Popperian argument that ``counterfactual theories are essentially metaphysical,'' and it has inspired Przeworski (2006) to ask whether ``the science of comparative politics is possible?''

Robins and Greenland (2000) present a potential Popparian counterargument to Dawid's criticism of counterfactuals; according to their interpretation, Popper required that a theory must make \textit{some} testable predictions that can be confronted with the data, all the while still relying on a core of untestable assumptions (one could argue that the rationality assumption falls under this category as well). \\

%The relevant question is thus whether is makes sense to study the \textit{causal} effect of institutions on a given outcome. Opsummering: bla bla bla when it comes to institutions does it make sense to talk of causality? is being black a causal??\\

Perhaps even more severe than the assumption of counterfactuals are the assumptions of random assignment of treatment and identical treatment effects. To highlight the first problem we return to consider the effect of our institutional treatment $D_i$ on our imagined population. Note that a naive comparison of the expected value of the treated and the non-treated produces neither ATET nor ATE
\begin{align}
E[Y_{1i}|D_i=1 ]-E[Y_{0i}|D_i=0 ]&=E[Y_{1i}|D_i=1 ]-E[Y_{0i}|D_i=1 ]\\\nonumber &+E[Y_{0i}|D_i=1 ]-E[Y_{0i}|D_i=0 ].
\end{align}
Equation (3) illustrates that a naive comparison between the two groups produces the sum of ATET and the well known selection bias. Note the counterfactual nature of the selection bias; we theoretically compare the expected value of the treated had they not received treatment with the expected value of the non-treated. Selection bias can occur if assignment is non-random - for example if selection is done based on observables - or units self-select into different categories. In our case of institutions one needs to consider whether assignment of institutions is non-random or whether units self-select on the basis of treatment. As an example, consider the assignment of democracies across nations. Could the assignment of democracy co-vary with unobservables such as culture? Furthermore, one needs to worry that nations self-select; that is, that particular nations transit to democracy whereas other abstain. 

Random assignment of $D_i$ solves the selection bias problem. Intuitively, random assignment guarantees that treatment is assigned independently  of observed as well as unobserved covariates. If assignment is truly random, $D_i$ can be said to be independent of potential outcomes: $E[Y_{ji}]=E[Y_{ji}|D_i=j]$ for $j=0,1$. Note that we are now able to rewrite (2) 
\begin{align}
E[Y_{1i}|D_i=1]-E[Y_{0i}|D_i=0 ]&=E[Y_{1i}|D_i=1 ]-E[Y_{0i}|D_i=1 ]\\\nonumber  &=E[Y_{1i}-Y_{0i}|D_i=1],
\end{align}
illustrating that under random assignment a comparison between the two groups produces both ATET and ATE. 

One problem outshines all other in analyzing the causal impact of institutions on growth: for obvious reasons, assignment of institutions is not done by the researcher, but by ``something else'' (following Haavelmo we continue by labelling this somewhat imprecisely as ``nature''), and only in extremely rare occasions can we expect assignment to be even approximately random. 

Researchers have been immensely creative in trying to overcome the non-random assignment of institutions, relying on econometric tools imported from labour economics such as IV, Differences in Differences, Matching and Regression Discontinuity. The most successful are perhaps AJR and Banerjee and Iyer. Both papers rely on IV, and in the latter case also on DID. However, the two papers differ fundamentally in one important aspect: Banerjee and Iyer study within country variation in India, whereas AJR look at cross country variation. I suspect this difference helps explain why the AJR paper has been subject to substantial criticism (even today AJR's methodology remain highly controversial), whereas the Banerjee and Iyer paper remains severely less criticized. This could reflect the fact that it is a lot more difficult to find valid cross country instruments when the assignment of institutions is non-random compared to looking only at within country variation. I return to this point in the concluding remarks. \\

So far we have dealt with causality from a purely abstract perspective. In reality, researchers rely on regression analysis to identify parameters of interest. A regression model corresponding to the treatment variable $D$ can be written as
\begin{equation}
Y_{it}=\alpha+\beta_iD_i+\epsilon_i,
\end{equation}
where $\beta_i=E[Y_{1i}|D_i=1]-E[Y_{1i}|D_i=0]$ captures the causal effect of $D_i$ on individual $i$ under random assignment. Note that in order to be able to estimate ATE we need to make an assumption of \textit{identical treatment effect}. That is, we assume that the effect of treatment is identical for individuals who share the same covariates. Mathematically: $\beta_i=\beta$ for all $i$. 

The identical treatment assumption effectively states that Saudi Arabia would have had  the same growth rate as Denmark, provided it had the same political institutions, the same per capita income, fewer Muslims, etc. It seems unrealistic, however, given that the institutional setting most likely affects the possible economic sectors that emerge and survive (as pointed out by Acemoglu et al. 2008), and the effect on growth of Saudi Arabia adopting a parliamentary voting system as in Denmark might be very different. \\

This section has attempted to illustrate that causal identification of institutions remains, to put it optimistically, extremely difficult and tedious. We operate in counterfactual territory where only few of our assumptions are directly testable. Furthermore, assignment of institutions is most likely never random (actually, as Przeworski (2006, p. 24) states ``the motor of history is endogeneity'') and the identical treatment effect remains untestable. So where do we go from here? I would argue that we should narrow our attention to within country studies where random experiments are possible, or where history has done the randomization for us. In this respect, I view Dell's and Berger's (2009) experimental approach promising. However, we need to humbly recognize that convincing evidence of causality might still be out of reach, and that so far results are at best indications. 

\section{}
In section 1, I focused on three different institutions that could affect growth. In this section, I present a sketch of two models: one model showing how a country can get trapped in bad property rights institutions and another model showing how a country can get trapped in bad contracting institutions.\footnote{Space did not allow for a sketch of a model of how a country can get trapped in bad conflict processing institutions. Such a model could build on interesting work by Xi (2011a, 2011b)} The first is inspired by a model by recent work by Besley and Persson (2009, 2010, 2011) and earlier work by Acemoglu (2006), whereas the latter is inspired by ongoing work by Vardy (2010). 

There is a well developed economic literature focusing on poverty traps (arbitrary examples include Rosenstein-Rodan 1943 and Bowles et al. 2006). Many of these models are build on the assumption that market imperfections such as externalities or incomplete information can lead to existence of multiple equilibria that can be Pareto-ranked. Several of these models rely on the following intuition: because of market failures, a country can get trapped in a Pareto inferior equilibrium, a poverty trap. Getting out of this trap requires coordinated activities among the agents in the economy that the market itself cannot deliver. 

Besley and Persson attempt to endogenize a state's capacity to tax its citizens or enforce property rights. In the following I focus only on the latter. One novel aspect of their models is that the capacity to enforce property rights is modelled as an investment decision. The building blocks consists of the following: assume that society consists of two groups who both live for two periods (today, tomorrow). Each agent in a given group has identical preferences. Assume that an individual from one of the groups is decision maker today, but that there is an exogenous probability that power shifts tomorrow. Furthermore, assume that there is an exogenous risk that there is a conflict of interest between the two groups tomorrow. The individual has to decide on an investment today in capacity to enforce property rights tomorrow. Can a situation with bad property rights institutions exist in this setup? Besley and Persson show the answer to be yes (under very specific conditions). It turns out that if the risk of conflict is high the incumbent agent will want to underinvest in legal protection for the other group. Because aggregate production is increasing in the protection of property rights this lowers growth in the economy. Thus, an economy with very heterogeneous groups is more likely to get stuck in a bad property rights institutional trap compared to an economy where the risk of conflict between the two groups is low. \\

We now turn our attention to a model of contractual institutional traps as in Vardy. The basic intuition of the model is the following. The economy consists of a finite number of firms, all of which has to decide whether to produce a good in-house, or buy it from another firm. Thus, it faces a trade-off between the gains from specialization and the costs of transacting in the market. Aggregate production depends positively on the rate of specialization in the economy. On the firm level however, increasing transaction costs makes specialization relatively less profitable. Now consider a country with bad initial contractual institutions, and thus high transaction costs. The firms in this economy will find it more attractive to rely on in-house production of several goods, and the economy will be characterized  by vertically integrated product chains and non-specialized firms that trade very little with each other. We can now evaluate the effect of a marginal increase in the institutional quality; since only very few firms interact with each other, a marginal increase will benefit the economy relatively little. 

We now compare this outcome with a country with good initial contractual institutions. By contrast, firms in this economy will be highly specialized and transact a lot with other firms in the economy. Increasing marginal institutional quality results in large gains in this economy. Hence, countries that are blessed with good initial conditions will benefit from an increase in institutional quality, creating an incentive for them to invest (assuming they solve their collective action problem) in even better institutions, whereas the opposite is true for countries starting out with bad initial conditions. Thus, the contractual institutional trap is created by the fact that returns are increasing for countries with good initial conditions, but decreasing for countries starting out with weak institutions. \\

In this section I have outlined two ways of thinking about institutional traps. Even though I recognize that both approaches offer interesting models of thinking about institutions, I do believe they run the risk of substituting empirical predictions for mathematical rigour. I am not confident that the marginal returns to theoretical papers on institutions is likely to be high at the moment. I have more faith in the approach offered by Romer trying to unbundle institutions into sets of rules and meta-rules, as I believe this approach is more likely to offer testable predictions. 




\clearpage
\begin{thebibliography}{9} %

\singlespacing

\bibitem[] {a} Acemoglu, D., S. Johnson, J. Robinson. and P. Yared. 2008. Income and Democracy. \textit{American Economic Review}, 98, pp. 808-842. 


\bibitem[] {a} Angrist. J. D. and J. S. Pischke. 2009. \textit{Mostly Harmless Econometrics}, Princeton: Princeton University Press.

\bibitem[] {a} Besley, T. and T. Persson. 2011. \textit{Pillars of Prosperity},  Princeton: Princeton University Press. 

\bibitem[] {a} Bowles, S. S. N. Durlauf. and K. R. Hoff. \textit{Poverty Traps}, Russell Sage Foundation. 


\bibitem[] {a} Casey, K. R. Glennerster. and T. Miguel. 2011. Reshaping Institutions: Evidence 
on External Aid and Local Collective Action, NBER working paper 17012. 

\bibitem[] {a} Dawid, A. P. 2000. Causal Inference Without Counterfactuals, \textit{Journal of the American Statistical Association}, 95, pp. 407-424.

\bibitem[] {a} Fearon, J. M. Humphreys. and J. Weinstein. 2011. Development 
Assistance, Institution Building, and Social Cohesion after Civil War: Evidence from a 
Field Experiment in Liberia, \textit{American Economic Review: Papers \& Proceedings}.

\bibitem[] {a} Jones, E. L. 1981. \textit{The European Miracle: Environments, Economies, and Geopolitics in the 
History of Europe and Asia}, Cambridge: Cambridge University Press.


\bibitem[] {a} Jones, C. I. and P. M. Romer (2009). The New Kaldor Facts: Ideas, Institutions, Population, and Human Capital, NBER working paper 15094. 



\bibitem[] {a} Przeworski, A. 2006. Is the Science of Comparative Politics Possible?, C. Boix. and S. C. Stokes (eds.), \textit{Oxford Handbook of Comparative Politics}.

\bibitem[] {a} Robins, J. M. and S. Greenland. 2000. Comment on Causal Inference Without Counterfactuals, \textit{Journal of the American Statistical Association}, 95, pp. 431-435.

\bibitem[] {a} Romer, P. M. 2010. Which Parts of Globalization Matter for Catch-Up Growth?  NBER working paper 15755. 

\bibitem[] {a} Rosenstein-Rodan, P. N. 1943. Problems of Industrialisation of  Eastern and South-eastern Europe, \textit{The Economic Journal}, pp. 202-211. 


\bibitem[] {a} Vardy, F. 2010. Institutional Traps, U.C. Berkeley working paper. 

\bibitem[] {a} Williamson, O. 1975. \textit{Markets and Hierarchies.}, New York: Free Press. 

\bibitem[] {a} -------. 1985. \textit{The  economic  institutions  of  capitalism}. New  York: Free  Press.

\bibitem[] {a} Xi, T. 2011a. A Theory of Confict Management in Divided Societies Under
Majority Rule and Power-Sharing, working paper. 

\bibitem[] {a} -------. 2011b. Constitutional Mediation and Political Compromise in Repeated
Conflicts, working paper. 


\end{thebibliography}

\end{document}
